\chapter{Introduction}
\label{chap:introduction}

In this course we will learn how to configure several different firewalls.
We will cover the basic configuration for management of the device, configure firewall rules, and set up IPsec VPN tunnels to connect different sites or offices securily together.



\section{What is a firewall?}
A firewall is a \emph{network security system} that monitors and controls incoming and outgoing network traffic based on predetermined security rules.
A firewall typically establishes a barrier between a trusted network and an untrusted network, such as the Internet.
The device offers a lot more features, however, such as a network address translator (NAT), intrusion prevention system (IPS),VPN concentrator, and more.

A classic firewall is a router, connecting different networks together.
It can filter traffic based on IP addresses and layer 4 protocols (e.g.\ TCP and UDP).
Nowadays firewalls can perform deep packet inspection and filter up to the appliction layer.
These devices are called next-generation firewalls (see page~\pageref{sec:intro-ngfw}).

It is not always convenient to introduce such a firewall into the network.
As the firewall is a layer~3 hop in the network, it is necessary to introduce a new network and assign new IP addresses to certain devices in order to create an additional hop for the firewall.
A possible solution is to configure the firewall as ``bump in the wire,'' the firewall interfaces do not get IP addresses and the firewall remains invisible to other devices in the network.

\Vref{fig:simple-lan} depicts a simple network setup with a local area network (203.0.113.0/24) connecting to the Internet via a router.
The link from the router to the router of the service provides uses addresses from the range 192.0.2.0/29.
When we introduce a firewall to this setup (see \cref{fig:simple-lan-fw}) we need to add an extra address range (192.0.2.8/29).
The router needs to be configured with a new IP address on the inside interface.


\begin{figure}
\begin{minipage}[b]{.47\textwidth}
\documentclass[tikz]{standalone}
\usepackage{ifthen}
\usepackage{contour}
\input{../tex/colors-mdt}
\input{../pgf/tikz-pgf}
\input{../pgf/layers}
\input{../pgf/connections}
\input{../pgf/nodes}

\begin{document}
\begin{tikzpicture}[framed]

\def\iconset{basic}
\hidelabels

\nodecloud{Internet}{0,2}{Internet}{}
\noderouter{router}{0,0}{Gateway}{}
\connect{Internet}{}{router}{}

\draw[{Square[open]}-{Square[open]},thick] (-1.5,-1.5) -- (1.5,-1.5);
\draw (router) -- (0,-1.5);

\node at (1.5,.7) {192.0.2.0/29};
\node at (0,-2) {203.0.113.0/24};

%\showhostnamesinterfaces
%\connect{switch}{fa0/3}{pc 3}{eth0}

\end{tikzpicture}
\end{document}

\caption{A simple setup with a router connecting the local network to the Internet}
\label{fig:simple-lan}
\end{minipage}
\hfill
\begin{minipage}[b]{.47\textwidth}
\documentclass[tikz]{standalone}
\usepackage{ifthen}
\usepackage{contour}
\input{../tex/colors-mdt}
\input{../pgf/tikz-pgf}
\input{../pgf/layers}
\input{../pgf/connections}
\input{../pgf/nodes}

\begin{document}
\begin{tikzpicture}[framed]

\def\iconset{basic}
\hidelabels

\nodecloud{Internet}{0,4}{Internet}{}
\noderouter{router}{0,2}{Gateway}{}
\nodefirewall{firewall}{0,0}{Firewall}{}
\connect{Internet}{}{router}{}
\connect{router}{}{firewall}{}

\draw[{Square[open]}-{Square[open]},thick] (-1.5,-1.5) -- (1.5,-1.5);
\draw (firewall) -- (0,-1.5);

\node at (1.5,2.7) {192.0.2.0/29};
\node at (1.5,1) {192.0.2.8/29};
\node at (0,-2) {203.0.113.0/24};

%\showhostnamesinterfaces
%\connect{switch}{fa0/3}{pc 3}{eth0}

\end{tikzpicture}
\end{document}

\caption{The same setup as \cref{fig:simple-lan} but after introducing a firewall to the setup}
\label{fig:simple-lan-fw}
\end{minipage}
\end{figure}


\section{Next-generation firewalls}
\label{sec:intro-ngfw}

% https://www.juniper.net/documentation/en_US/learn-about/LA_FIrewallEvolution.pdf

First generation firewalls were relatively simple filter systems called \emph{packet filter firewalls}, but they made today's highly complex security technology for computer networks possible.
Packet filter firewalls, also referred to as \emph{stateless firewalls}, filtered out and dropped traffic based on filtering rules.
Packet filter firewalls did not maintain connection state.
That is, a packet was processed as an atomic unit without regard to related packets.
Packet filter firewalls were deployed largely on routers and switches.

Typically, packet processing was performed at layers~3 and~4 by matching the header fields of the packet against four tuples: who sent the packet and who was its intended recipient (source and destination IP addresses), the ports (source and destination ports), and the protocol used to transport the traffic (e.g.~TCP and ICMP).
No higher layer information was taken into account.

The main weakness of packet filter firewalls was that hackers could craft packets to pass through the filters taking advantage of the lack of state.
When packet filter firewalls were first used, operating system stacks were vulnerable and a single packet could crash the system, an event that rarely occurs today.
Packet filter firewalls allowed all web-based traffic, including web-based attacks, to pass through the firewall.
These firewalls could not differentiate between valid return packets and imposter return packets.
How to handle these and other similar problems set the stage for future firewall development.

Packet filter firewalls were followed not long afterwards by \emph{stateful firewalls}.
These second generation firewalls has the same capabilities as packet filter firewalls, but they monitored and stored the session and connection state.
They associated related packets in a flow based on source and destination IP addresses, source and destination ports, and the protocol used.
If a packet matched this information bi-directionally, then it belonged to the flow.

Stateful firewalls addressed the packet filter firewall problem of not being able to determine if a return packet was from a legitimate connection, but the problem of not being able to differentiate good web traffic from bad remained.
What was needed were firewall features that could detect and block web attacks, and they followed soon afterward.




\section{Virtual private networks}
\label{sec:intro-vpn}



\section{Firewall vendors and features}
\label{sec:intro-vendors}
