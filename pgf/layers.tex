%
% layers.tex
%
% This document contains code to hide a PGF layer if so desired,
% e.g. you can hide the layer with all the labels on it.
%

% Define the layers to be used
\pgfdeclarelayer{connections}
\pgfdeclarelayer{interfaces}
\pgfdeclarelayer{hostnames}
\pgfdeclarelayer{labels}
\pgfdeclarelayer{arrows}

% Make the background white so we can hide the labels by swapping the order of the layers
\tikzset{show background rectangle,background rectangle/.style={rounded corners,fill=white}}

% \showalllayers - Show the nodes, connections, and detailed labels, including interface labels
% The left most layer is the bottom layer; everything left from the background will be hidden.
\newcommand\showalllayers{      \pgfsetlayers{hostnames,arrows,background,connections,main,interfaces,labels}}
\newcommand\showarrows{         \pgfsetlayers{hostnames,background,connections,main,interfaces,labels,arrows}}
% \hidelabels - Show only the nodes and the connections
\newcommand\hidelabels{         \pgfsetlayers{hostnames,arrows,interfaces,labels,background,connections,main}}
% \hideinterfaces - Hide only the interface labels
\newcommand\hideinterfaces{     \pgfsetlayers{hostnames,arrows,interfaces,background,connections,main,labels}}
% \showonlyhostnames - Hide everything but show the hostnames only
\newcommand\showonlyhostnames{  \pgfsetlayers{arrows,interfaces,labels,background,connections,main,hostnames}}
\newcommand\showhostnamesinterfaces{ \pgfsetlayers{arrows,labels,background,connections,main,hostnames,interfaces}}
% Default setting is to show all layers
\showalllayers